						INTEGER PROGRAMMING

						Branch & Bound Algorithm



						Queueing Theory
								|
								|
				___________ | ________
				|							|

				Queue                         Server
									1. Parallel Processing.
									2. Serial/Batch Processing.

To study the queue we need to understand the queue characteristics.
QT is about understanding the Queue Characteristics.

Characteristics:
1.
2.

Queue Characteristics can be modelled using:
1. Exponential Distribution.
2. Poisson Distribution.

Random Variable
|
|_____>  Discrete RV.
|_____> Continuous RV.

Probability Density Function. (PDF) (for Continuous RV)
Probability Mass Function. (PMF) (for Discrete RV)

Cummulative Distribution Function. (CDF)

Elements of Queueing Theory

Principal Actors
--> Customer
--> Server (A facility which gives service the customer)

Interarrival time : 

Queue:
1. Finite.
2. Infinite.
Queue Discipline:
* FCFS (First Come First Serve)
* LCFS (Last Come First Serve)
* SIRO
* Order of Priority.

Queue Behaviour:
* Jockeying.
* Balking.
* Reneging.

Design of service facility may include the following points:
1. Parallel Servers.
2. Series Servers.
3. Servers in Network.

Task: How is Parallel Servers different from Servers in Network?


Waiting Line Process (Queueing Process)
1. Customers
2. Service.

Queueing Process can be of Single Server or Multiple Server.
1. Single Server --> only 1 service station for the entire queue. (Movie ticket booking, food market).

2. Multiple Server --> Gas station with multiple serving station.

The process of waiting line is characterized by the rate at which the customer arrives. 

i> Arrival Rate specifies the average number of customers per time period.
ii> The service rate specifies the average number of customers served during a time period.

What we need to do?
Our task is to determine the probability of ‘n’arrivals being observed during a time interval of length ‘t’.

The probability is based on the following pre-assumptions:
1. The Probability that an arrival is observed during a small time interval is proportional to the length of the interval.

2. The probability of two or more arrivals in such a small interval is zero.

3. The no. of arrivals in any time interval is independent of the number in non-overlapping time interval.

P( ‘n’ customers during period ‘t’) = the probability that ‘n’ arrivals will be observed in a time interval of length ‘t’.

P(‘n’ arrivals in time ‘t’) = lambda * exp(-lambda*t)
lambda --> Avg. rate of arrival.

* This situation in QT(Queueing Theory) is called poisson’s arrival.
* Since the arrivals alone are considered, not the departure, it is called Pure Birth Process.

‘n’ --> Random Variable.
‘n’ --> No. Of arrivals.

Interarrival Time
The time between successive arrivals is called interarrival time.
When the no. Of arrivals in a given time interval has poisson distribution, the interarrival time has exponential distribution.

f(t) = 
P(t <= T) = 1-exp(-lambda*T)
P(t>T) = exp(-lambda*T)

Ex:
1. One arrival comes in every 15 mins.
	Average arrival time = 4 /hr
	Avg. Interarrival time = 0.25 hrs

2. In a factory the machine breaks down and requires service a poisson distribution, at the avg. Rate of 4/day. What is the probability that exactly 6 machine break down in 2 days.

P(n,t) = (((lambda*t)^n)*exp(-lambda*t))/n!

3. On an avg. 6 customers arrive in a coffee shop per hour. Determine the probability that exactly 2 customers will reach in a 30 min period.

Lambda = 6, t= 30 mins = 0.5 hrs
P(6,2) = 0.224.

traffic intensity (rho) = (lambda/mu) = Avg. Arrival time / Avg. Service time.
Rho < 1
 
C = No. Of parallel service channels.

Markov modelling (M/M/1 model) (alpha / FIFO)

Assumptions for the M/M/1 Modelling:
i> Arrivals follow poisson’s distribution, with lambda.
Ii> Service time follows exponential distribution with mu, Average service rate.
Iii> Arrivals are infinite populations (alpha = inf)
iv> Customers are served in FIFO.
v> There is only a single server.

For the system to work, No. Of arrivals < Avg. Service rate.

Rho = lambda/mu <---- Probability of time.

Probability that system is idle = Probability that there are no customers in the system.

P0 = 1 – rho.
Probablity of having exactly one customer in the system.

Pn = rhon * P0  = 

* The expected number of customers in the system:

Ls = ∑ (1 -> alpha (inf) ) n * Pn
= rho/(1-rho) = lambda/(mu-lambda)

* The expected number of customers in the system:

Lq = ∑ (1 > alpha) (n-1)*Pn 


Lq = rho2 / (1 – rho) = lambda2 /(mu* (mu – lambda))

* Average waiting time in the system

Ws = (Avg. Time b/w arrivals)*()


* Avg. Waiting time in the queue

Wq = Avg. Time b/w arrivals * Lq  = (1/ lambda )* Lq 
Wq  = 

Ex. Mean Arrival Rate = 1 customer/4 mins
	 Mean Service Time = 2.5 mins

1. Average customers in the system.
2. Average Queue length.
3. The time taken by the customer in the system.
4. Average time a customer waits before being served.

1. Ls = 1.66 customer/hr
2. Lq = 1.04 customers/hr
3. Ws = 6.66 mins
4. Wq = 4.16 mins



						Kendall Notation

a = Arrival distribution
b = Service time (has exponential distribution)
c = no. Of service channels (Servers)
d = Maximum number of customers allowed in system
e = queue discipline (FIFO / FCFS)

[ (a/b/c) = (d/e)]
[(M/M/1) = (alpha/FIFO)]
* For M/M/1 queueing system there are exact formulas, which state that the system has reached a steady state.

A steady state for a system means that the system has been running long enough ,so as to settle down into some kind of equilibrium position.

M/M/s Queueing System

lambda = mean arrival rate.
S = No. Of servers (always >= 1)

rho (utilization factor/traffic intensity)  = if S increases rho decreases.

Rho = lambda /(mu * S)

To have a steady state
lambda < (mu * S)  i.e.  rho < 1

Rate Diagrams
General Rate Diagram

Rate diagram for M/M/s model (lambda fixed mu changes).


Po = Probability that there are no customers.
Po = 1 – rho

Lq = (Po(lambda/mu)^n rho)/(s!*(1-rho)^2)
 

		M/M/s//K Queueing Model ( With finite population ‘K’ )
			( Finite Queue variation of M/M/s )



Since, there are a finite number of states, the steady state equation hold even if rho > 1.

			M/M/s///N Queueing Model
				( Finite Calling Population variation of M/M/s )

			(Population is subset of the capacity)








						Non-Linear Programming

Non-Linear Programming Introduction

1. One-dimensional minimization methods.

Fibonacci Method

Fibonacci method can be used to find the minimum of a function of one variable even if the function is not continuous.
It is also similar to dichotomous method but the two are not the same i.e. is applicable also for  non continuous functions also.

Limitations:
i> The initial interval of uncertainty, in which the optimum lies, has to be known.
ii> The function to be optimized has to be unimodal (in an interval only 1 minimum can lie) in the initial interval of uncertainty.
Iii> The exact optimum cannot be located in this method. Only an interval known as the final interval of uncertainty can be made as small as desired by using more computation  
iv> The number of function evaluations to be used in the search or the resolution required has to be specified beforehand.

F0 = F1 = 1
Fn = Fn-1 + Fn-2  n=2,3,4...
L0 = [a, b]
L2* = (Fn-2 / Fn)*L0
x1 = a + L2* = a + (Fn-2 / Fn)*L0
x2 = b – L2*      = a + (Fn-2 / Fn)*L0
L2 = L0 – L2* = L0 - (Fn-2 / Fn)*L0 = (Fn-1 / Fn )*L0
L2* = (Fn-2 / Fn )*L0 = (Fn-2 / Fn-1 )* L2
L3*  = (Fn-3 / Fn )*L0 = (Fn-3 / Fn-1)* L2
L3 = L2 – L3* = L2 – (Fn-3 / Fn-1)*L2 = (Fn-2 / Fn-1)*L2

After jth iteration:
(Start j from 3)
Lj = (Fn-j / Fn-(j-2))*Lj-1
Lj = (Fn-(j-1)/Fn)*L0
Reduction Ratio (Ln / L0):




Fibonacci Method:

n         Fn           Ln / L0
0 		1			1
1		1			1
2		2			0.5
3		3   			0.333
4		4			0.2
5		8			0.125
6		13			0.07692
7		21			0.04762
8		34			0.02941
9		55			0.01818
10		89			0.01124

Ex. L0 = [0, 3]                             L0  = [0, 3]		
	L2  = [1, 1.5]					 L2 = [0, 1.8462]
	(L2 / L0 ) = 1/6				(L2 / L0 ) = 


Ex. Min f(x) = 0.65 – [0.75/(1+x2)] – 0.65*x*tan-1 (x) by fibonacci method using n = 6.

L2* = (Fn-2 / Fn )*L0 = (F4 / F6)*L0 = 5/13 * 3 = 1.153846

fa = f(0) =          fb = f(b) = 

x1 = a + L2*  = 0 + 1.153846
x2 = b – L2* = 3 – 1.153846 = 1.846154
f1 = f(x1) = -0.207270 , f2 = f(x2) = -0.1158

Using unimodality theorem,
L2 : [0, x2]
L2* : [x2, 3]
x3 = a + x2  - x1 = 0 + (1.846154 – 1) = 0.692308
f3 = f(x3) = -0.291364
Since, f1 > f3 
L3 = L2 – L3* = [0, x1] 
x4 = a + (x1 – x3) = 0.461538
f4 = 0.30981

f4 < f3 

L4 = [0,x3] = L3 – L4* = [0, x1] – [x3, x1]
			= [0,x3]

x5  = a + (x3 – x4) = 0.23077
f5 = -0.263678
f5 > f4

a = 0			fa = -0.1
x5 = 0.23077			f5 = -0.2637
x4 = 0.4615			f4  = -0.309811
x3 = 0.6923			f3  = -0.2914

L5 = L4 - L5*  = [0, x3] – [0,x5] = [x5 , x3]
x6   = x5 + (x5 – x4) = 0.461540
f6 = -0.309810

Since, f6 > f4
x5  = 0.23077 		f5 = -0.2636    
x4 = 0.461538		f4 = -0.3098
x3 = 0.69231		f3  = -0.29136
x6 = 0.461540 		f6 = -0.309810

L6  = L5 – L6* = [x5, x6]
L6 / L0  = 0.076923

L6 / L0 = 1/F6 = 0.07692


			Golden Search Method



Golden Search Method is based on the Golden Ratio.

Ex. 1>  A line segment is divided into 2 unequal parts:

<------------------------------------->
|<--------- a ---------->|<---- b --->|

(a+b)/a = a/b ( =r ) . For what value of  ‘a’ can this happen?

b*(a+b) = a2

for b = 1								for b = 2
a2 -a -1 = 0
a = (1+/- (5)1/2)/2

a/b = (1 + (5)1/2)/2 = 1.6180339



From Fibonacci Sequence:

N  	2	  	3  		4  		5  		6  		7  		8  		9 		10		∞
(Fn-1/Fn)        0.5  0.667    0.6    	0.625 	0.61 	0.62	0.617	0.618 			0.618


Now from fibonacci method, for large N (N --> ∞)

L2 = lim(N --> ∞) Fn-2

At Kth  iteration:
Lk = lim (N --> ∞) (Fn-1/Fn)k-1 * L0
Now, from the relation:
Fn = Fn-1 + Fn-2
Fn/Fn-1 = 1 + (Fn-2/Fn-1)
(Fn/Fn-1) = 1 + (Fn-1/Fn) -------------------(2)

Let’s define
r = lim (N --> ∞) (Fn/Fn-1)   ----------------------(3)
r = 1 + 1/r
=> r2 - r -1 = 0
=> r = 1.618034        -------------------(4)

using (3) and (4) in (1):
Lk = (1/r)k-1 L0
      = (1/1.618034)k-1 * L0

L2* = lim (N --> ∞) (Fn-2/Fn)*L0
        = lim (N --> ∞) (Fn-1/Fn)2 * L0
        = (1/r)2 * L0

Ex. Min f(x) = 0.65 – 0.75/(1+x2) – 0.65*x*tan-1(1/x)
[0,3] , n = 6

Soln. L0 = [0, 3]

L2* = 0.382*L0  = 0.382 * (3 – 0) = 1.1460
x1 = a + L2*  = 0 + 1.1460 = 1.1460			f1  = -0.208654
x2  = b – L2* = 	3 – 1.460	 = 1.8540			f2  = -0.115124
fa = -0.1
fb = -0.0524

By the unimodality assumption of the function f(x), the interval [x2,3] is discarded and [0,x2] is the new interval of uncertainty.

L2 = L0 – L2*  = [0, 3] – [x2, 3] = [0, x2]
x3  = a + (x2 – x1) = 0 + (1.8540 – 1.1460) = 0.7080
f3  = -0.288943

Since f3 < f1 , by unimodality assumption of f(x), interval [x1.x2] to be discarded, and [0,x1] is the new interval of uncertainty.

L3 = L2 - L3*  = [0,x2] – [x1, x2] = [0,x1]
x4 = a + (x1 – x3) = 0 + (1.1460 – 1.)
x4 = 0.4380                x5 = 0.2700			x6 = 
f4 = -0.308951			f5 = -0.278434


Min/ Max function f(x,y,z) subject to a constraint g(x,y,z)=c.

Langrangian multiplier method for constraint optimisation.


Constrained Optimization with equality Constraints

Ex: 2
Lagrangian Function

L(x,y,z,λ1,λ2) = f(x,y,z) - ∑(i = 1 -> m) λi * gi (x,y,z)


1. 2*x – 2*λ1 = 0 		=> x = λ1
2. 2 + λ1 – λ2  = 0		=> λ2 = λ1 + 2
3. -2*z – λ2 = 0		=> z = -λ2/2
4. 2*x – y = 0			=> 2x + z = 0
5. y + z = 0

Solving the above equations:
λ1 = 2/3 , λ2 = 8/3 , x = 2/3 , y = 4/3 , z = -4/3

The maximum point is (2/3,4/3,-4/3)
and the maximum value = f(2/3,4/3,-4/3) = 4/9 + 8/3 - 16/9= 4/3

Ex : 3 Find optimal value of f(x,y,z) = x2 + y2 + z2 
		w.r.t the constraints:
1. x + 2*y + 3*z = 6
2. x + 3*y + 9*z = 9



					Convex and Concave Function

Convex Function:

Concave Function:


Neither Convex nor Concave function:







Dt. 01/05/2020
Constrained Optimization with Inequality Constraints

Max/Min f(x1,x2,....,xn)

S.T.
g1(x1,x2,...,xn) <= b1
g2(x1,x2,...,xn) <= b2
gm(x1,x2,...,xn) <= bm

x1,x2,....,xn >= 0

Kuhn – Tucker (KT) Conditions

a> If  1) is maximization problem and x_bar = (x1_bar,x2_bar,...,xn_bar) is an optimal solution then x_bar must satisfy the m constraints in 1) are g1, g2, ... gm and there must exist multipliers lambda_1, lambda_2, ..., lambda_m satisfying:

df/dx_j - ∑(t = 1-> m) λi * dg_4(x_bar)/dx_j = 0 (j=1,2,3,...,n)
λ_i [b_i – g_i(x_bar)] = 0
λ_i >= 0 (i=1,2,3,....,m)

b> If 1) is a minimization problem and x_bar=(x1_bar,x2_bar,...xn_nar) is an optimal solution of 1), then x_bar must satisfy the m constraints in 1)  and there must exist multipliers lambda_1, lambda_2, ...,lambda_m satisfy.

df(x_bar)/dx_j + ∑(t=1->m) dg_i(x_bar)/dx_j = 0


Ex. Max z = x_1(30-x_1)+x_2(50-2*x_2)-3*x_1-5*x_2-10*x_3

S.T.:
	x_1 + x_2 <= x_3
	x_3 <= 17.25

f(x_1,x_2,x_3) = x_1(30-x_1) + x_2(50-2*x_2) - 3*x_1 – 5*x_2 - 10*x_3


The KT – Conditions are:
df/dx_1 - lamda_1*




 
